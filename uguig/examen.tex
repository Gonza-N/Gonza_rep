\documentclass[
  11pt,
  letterpaper,
  % addpoints,
  answers
]{exam}

\usepackage{../tarea}

\begin{document}
\begin{minipage}{0.42\textwidth}
    \includegraphics[width=\textwidth]{../fcfm_die}
\end{minipage}
\begin{minipage}{0.53\textwidth}
\begin{center} 
\large\textbf{} \\
\normalsize Prof.~Gonzalo Narváez.
\end{center}
\end{minipage}

\vspace{0.5cm}
\noindent
\begin{questions}



\question \label{q:lt} Se tiene la siguiente línea de transmisión con distintos tipos de adaptadores, como se ve en la \cref{fig:transmission_line}. Todo el sistema tiene una impedancia característica de $Z_{1} \neq 0$. Se quiere adaptar la línea desde el puerto 1. Para esto:

  \begin{parts}
      \part[3]{ \label{prob:lt_a} Demuestre la formula para el adaptador $\frac{\lambda}{4}$.}

      \part[3]{Demuestre la formula para el adaptador $\frac{\lambda}{2}$}

      \part[4] Obtenga $Z_{in}$ adaptando completamente la LT de la \cref{fig:transmission_line}.  
  \end{parts}

  \begin{center}
    \begin{tikzpicture}
      \edef\width{1.5};
      \edef\wavel{19};

      \edef\angle{50};
      \coordinate (O) at (0, 0);
      \coordinate (A) at (0, {\width/2});
      \coordinate (B) at (0, -{\width/2});

      \draw[] (A) -- ++ ({\wavel/2}, 0) node[circle, draw, fill=black, inner sep=1pt] {} coordinate (Aext) -- ++ ({\wavel/4}, 0) coordinate (A1);
      \draw[] (B) -- ++ ({\wavel/2}, 0) node[circle, draw, fill=black, inner sep=1pt] {} coordinate (Bext) -- ++ ({\wavel/4}, 0) coordinate (A2);

      \node[red, draw] (Z2) at ($(A1)!0.5!(A2)$) {$Z_2$};
      \draw[red, thick] (A1) -- (Z2.north);
      \draw[red, thick] (A2) -- (Z2.south);
      \draw[red, thick] (A1) -- (Aext);
      \draw[red, thick] (A2) -- (Bext);
    
      

      \draw[dashed] (Aext) -- (Bext) node[midway, fill=white] {$Z_\text{in}^\prime$};

      \draw[<->] ($(A) + (0, .5)$) -- ($(Aext) + (0, .5)$) node[midway,fill=white] {$\lambda/2$};
      \draw[<->] ($(Aext) + (0, .5)$) -- ($(A1) + (0, .5)$) node[pos=.5,fill=white] {$\lambda/4$};
    


      \draw[stealth-] ($(A)!.5!(B)$) -| ++ (-.5, -1) node[below] {$Z_\text{in}$};

      \node[below] at (A) {Puerto 1};
    \end{tikzpicture}
    \captionof{figure}{Línea de transmisión correspondiente a \cref{q:lt}. Todo el sistema posee una impedancia intrínseca de $Z_1$. Se definen una zona en {\color{red}\textbf{rojo}}, con largo $\lambda/4$ y carga $Z_2  \neq 0$ y una zona en negro con largo $\lambda /2$.}
    \label{fig:transmission_line}
  \end{center}







\end{questions}
\end{document}
