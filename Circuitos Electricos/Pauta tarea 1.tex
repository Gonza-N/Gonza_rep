\documentclass[
  11pt,
  letterpaper,
  % addpoints,
  answers
]{exam}

\usepackage{../tarea}

\begin{document}
\begin{minipage}{0.42\textwidth}
    \includegraphics[width=\textwidth]{../fcfm_die}
\end{minipage}
\begin{minipage}{0.53\textwidth}
\begin{center} 
\large\textbf{Electromagnetismo Aplicado} (EL3103) \\
\large\textbf{Pauta Tarea 1} \\
\normalsize Prof.~Pablo Medina\\
\normalsize Prof.~Fernando San Martin\\
\normalsize Ayudantes: Gonzalo Narváez.~Victor Ramírez. 
\end{center}
\end{minipage}

\vspace{0.5cm}
\noindent
\vspace{.85cm}

\begin{questions}

\question{Pauta de la tarea 1}
\begin{parts}
    \part{Calcular el determinante de las siguientes matrices:}
\[
\begin{bmatrix}
    \frac{1}{sC} + R_1 & \omega \\
    \omega & R_2 + R_3
\end{bmatrix},
\begin{bmatrix}
    5 & 7 & 2 \\
    1 & 2 & 0 \\
    8 & 4 & 0
\end{bmatrix}
\]
\begin{mdframed}
\textbf{Solución:}

% Aquí va la solución de las matrices
\end{mdframed}

\part{Resolver mediante regla de Cramer los siguientes sistemas de ecuaciones:}

\begin{equation*}
\begin{aligned}
    &2x + 5y = 4 \\
    -&3x + y = 0 \quad \text{Donde las incógnitas son \textit{\(x\) e \(y\)}}
\end{aligned}
\end{equation*}

\begin{mdframed}
\textbf{Solución:}

% Aquí va la solución del sistema de ecuaciones
\end{mdframed}

\part{Separar en fracciones parciales las siguientes expresiones en el ``dominio s''.}
% Aquí irían las expresiones
\begin{mdframed}
\textbf{Solución:}
% Aquí va la solución para fracciones parciales
\end{mdframed}

\part{Escribir en su forma polar el siguiente número complejo.}
\[
\frac{a + jb}{c - jd}
\]
\begin{mdframed}
\textbf{Solución:}
% Aquí va la solución
\end{mdframed}
\end{parts}
\end{questions}
\end{document}
