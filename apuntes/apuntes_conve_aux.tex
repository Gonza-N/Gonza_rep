\documentclass[
  11pt,
  letterpaper,
  % addpoints,
  answers
]{exam}

\usepackage{../tarea}

\begin{document}
\begin{minipage}{0.42\textwidth}
    \includegraphics[width=\textwidth]{../fcfm_die}
\end{minipage}
\begin{minipage}{0.53\textwidth}
\begin{center} 
\large\textbf{Astro}  \\
\large\textbf{Resumen astro c2} \\
\normalsize Prof.~\\
\normalsize Prof.~\\
\normalsize Ayudantes: 
\end{center}
\end{minipage}

\vspace{0.5cm}
\noindent
\vspace{.85cm}




\begin{parts}
    \part{Estudiar deslizamiento ``S''  S=0 cuando la prueba es de vacio y S = 1 cuando es la prueba de rotor bloqueado}
    \begin{equation}
        S = \frac{\omega_s - \omega_r}{\omega_s} = \frac{n_s - n_w}{n_s}
    \end{equation}
    \part{recordar usar voltaje fase neutro donde la conversión viene dada por:}
\end{parts}

\begin{equation}
    V_{fn} = \frac{V_{ff}}{\sqrt{3}}
\end{equation}

Formulas: para prueba de vacio S = 0
\begin{align}
    V_{fn}^2/P_{0} &= R_{p} \\
    X_{n} &= \frac{V_{fn}^2}{Q_{0}}\\
    Q_0 &= \sqrt{(V_{fn}I)^2 - P_0}\\
\end{align}
Fórmulas para prueba rotor bloqueado S = 1
\begin{align}
    v_1 + v_2 &= \frac{P_{cc}}{i_{cc}^2}\\
    x_1 + x_2 &= \frac{Q_{cc}}{i_{cc}^2}\\
    Q_{cc} &= \sqrt{(V_{cc}I-{cc})^2 - P_{cc}}\\
\end{align}








\end{document}