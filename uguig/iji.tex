\documentclass[
  11pt,
  letterpaper,
  % addpoints,
  answers
]{exam}

\usepackage{../tarea}

\begin{document}
\begin{minipage}{0.42\textwidth}
    \includegraphics[width=\textwidth]{../fcfm_die}
\end{minipage}
\begin{minipage}{0.53\textwidth}
\begin{center} 
\large\textbf{Clases Particulares} \\
\normalsize Prof.~Gonzalo Narváez.
\end{center}
\end{minipage}

\vspace{0.5cm}
\noindent

\begin{figure}[ht]
    \centering
    \begin{tikzpicture}
      % Línea de transmisión
      \draw[thick] (-4,1) -- (4.5,1);  % Línea superior
      \draw[thick] (-4,-1) -- (4.5,-1); % Línea inferior
      \draw[thick] (2,1.2) -- (4.5,1.2); % Línea derecha
      \draw[thick] (2,1.1) -- (2,1.3); % Línea izquierda
      \draw[thick] (4.5,1.1) -- (4.5,1.3); % Línea izquierda


      \draw[thick] (2,-0.6) -- (3.5,-0.6); % Línea izquierda
      \draw[thick] (2,-1.2) -- (2,-0.6); % Línea derecha
      
      % Generador 
      \draw (-4,-1) to[sV, l=$V_s$] (-4,1);  

      \draw[thick] (-1,1) -- (-1,0.5); % Línea vertical
      \draw[thick] (-1,-1) -- (-1,-0.5); % Línea vertical
      \draw (-1.5,0.5) rectangle (-0.5,-0.5); % Rectángulo
      \node at (-1,0) {$Z_L$}; % Etiqueta dentro del rectángulo

      \draw[thick, fill=blue!30] (3.5,-0.8) -- (3.5,-0.4) -- (4,-0.6) -- cycle;


      \filldraw[black] (4.5,-1) circle (2pt);
      \filldraw[black] (4.5,1) circle (2pt);

      \node at (-2.5,0) {$Z_0$};
      \node at (0.5,0) {$Z_0$};
      \node at (3.25,1.6) {$l = \frac{5\lambda}{12}$};
      \node at (2,-0.3) {$Z_{ca}$};


      
    
      
    \end{tikzpicture}
    \caption{Linea de transmisión en circuito abierto.}
    \label{fig:lt}
    \end{figure}
\end{document}
