\documentclass{article}
\usepackage{amsmath}
\usepackage{amsfonts}

\begin{document}

\section*{Demostración de que el Filtro Adaptado Maximiza la SNR}

El filtro adaptado (matched filter) está diseñado para maximizar la SNR en la salida al final del período de símbolo \( T \). Consideremos un sistema donde se transmite un símbolo \( s(t) \) en presencia de ruido blanco gaussiano \( n(t) \). La señal recibida \( r(t) \) es:

\begin{equation}
r(t) = s(t) + n(t)
\end{equation}

El filtro adaptado tiene una respuesta impulsional \( h(t) \) que es una versión invertida en el tiempo de la señal transmitida \( s(t) \):

\begin{equation}
h(t) = s(T - t)
\end{equation}

La salida del filtro adaptado en \( t = T \) es:

\begin{equation}
y(T) = \int_{0}^{T} r(\tau) h(T - \tau) d\tau = \int_{0}^{T} (s(\tau) + n(\tau)) s(\tau) d\tau
\end{equation}

Descomponiendo:

\begin{equation}
y(T) = \int_{0}^{T} s(\tau) s(\tau) d\tau + \int_{0}^{T} n(\tau) s(\tau) d\tau = E_s + n_s
\end{equation}

Donde:

\begin{equation}
E_s = \int_{0}^{T} [s(\tau)]^2 d\tau
\end{equation}

Y \( n_s \) es el ruido filtrado:

\begin{equation}
n_s = \int_{0}^{T} n(\tau) s(\tau) d\tau
\end{equation}

La varianza del ruido en la salida del filtro es:

\begin{equation}
\sigma_n^2 = \mathbb{E}[n_s^2] = \int_{0}^{T} \int_{0}^{T} \mathbb{E}[n(\tau) n(\lambda)] s(\tau) s(\lambda) d\tau d\lambda
\end{equation}

Dado que \( n(t) \) es ruido blanco gaussiano con densidad espectral de potencia \( N_0/2 \):

\begin{equation}
\mathbb{E}[n(\tau) n(\lambda)] = \frac{N_0}{2} \delta(\tau - \lambda)
\end{equation}

Donde \( \delta \) es la función delta de Dirac. Entonces:

\begin{equation}
\sigma_n^2 = \frac{N_0}{2} \int_{0}^{T} [s(\tau)]^2 d\tau = \frac{N_0}{2} E_s
\end{equation}

La potencia del ruido en la salida es:

\begin{equation}
N = \sigma_n^2 = \frac{N_0}{2} E_s
\end{equation}

Por lo tanto, la SNR en la salida del filtro adaptado es:

\begin{equation}
\text{SNR} = \frac{\text{Potencia de la señal}}{\text{Potencia del ruido}} = \frac{E_s^2}{\frac{N_0}{2} E_s} = \frac{2 E_s}{N_0}
\end{equation}

Esto demuestra que la SNR se maximiza al final del período de símbolo \( T \).

\end{document}
