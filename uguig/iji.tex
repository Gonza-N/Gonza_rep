\documentclass[
  11pt,
  letterpaper,
  % addpoints,
  answers
]{exam}

\usepackage{../tarea}

\begin{document}
\begin{minipage}{0.42\textwidth}
    \includegraphics[width=\textwidth]{../fcfm_die}
\end{minipage}
\begin{minipage}{0.53\textwidth}
\begin{center} 
\large\textbf{Clases Particulares} \\
\normalsize Prof.~Gonzalo Narváez.
\end{center}
\end{minipage}

\vspace{0.5cm}
\noindent
\begin{questions}

\question[10]{Sea una línea de transmisión visto en la \cref{fig:lt} de impedancia característica $Z_0 = 40\,\Omega$ e impedacia $Z_L = 50-50j$, se le adiciona una linea de transmisión de longitud $l = 5\lambda/12$ en circuito abierto.}
\begin{parts}
  \part[2]{Calcule la impedancia $Z_{ca}$ del circuito vista a una distancia $\frac{5\lambda}{12}$.}
  \part[2]{Calcule la impedancia equivalente en la linea de transmisión.}
  \part[4]{Determine la distancia $l$ con tal de adaptar la parte real de la admitancia equivalente y la distancia $l_s$ con tal de adaptar la parte imaginaria de la linea de transmisión tanto en corto circuito ($l_{s}^{cc}$) como en circuito abierto ($l_{s}^{ca}$).}
  \part[2]{Explique detalladamente el proceso de adaptación.}
\end{parts}

\begin{figure}[ht]
    \centering
    \begin{tikzpicture}
      % Línea de transmisión
      \draw[thick] (-4,1) -- (4.5,1); 
      \draw[thick] (-4,-1) -- (4.5,-1);

      % Línea de transmisión añadida
      \draw[thick] (4.5,1) -- (6.6,2.6); 
      \draw[thick] (4.5,-1) -- (6.6,0.6);

      %stub
      \draw[thick] (-2.5,1) -- (1.5,-2);
      \draw[thick] (-2.5,-1) -- (-1.1,-2);

      \draw[|<->|] (-2.5,1.4) -- (1,1.4) node[midway, fill=white] {$l$};

      \draw[|<->|] (-2.5-.3,-1-.3) -- (-1.1-.3,-2-.3) node[midway, rotate=-22, fill=white] {$l_s$};
      \draw[dashed] (-1.1,-2) -- (1.5,-2);
        
      %impedanciia Z_l
      \node[draw] (ZL) at (1,0) {$Z_l$};
      \draw[thick] (1,1) -- (ZL.north);
      \draw[thick] (1,-1) -- (ZL.south);

      % \draw (4.5,-0.8) -- (4.7,-1); % Línea derecha
      \draw[->] (4.5,-1+.4) -- ($(6.6,0.6)!.5!(4.5, -1) + (0,.4)$) node[midway, rotate={atan((2.9 - 1.3) / (6.3 - 4.2))}, above] {$Z_\text{ca}$};

      \draw (4.5,-1+.4) -- ++ (0, -1);

      % circuito abierto 
      \filldraw[black] (6.6,2.6) circle (2pt);
      \filldraw[black] (6.6,0.6) circle (2pt);

      \draw[|<->|] (4.5-.3,1+.3) -- (6.6-.3,2.6+.3) node[midway, rotate=40, fill=white] {$l_{ca} = \frac{5\lambda}{12}$};
      
      \node at (-2.7,0) {$Z_0$};
      \node at (3,0) {$Z_0$};
  
    \end{tikzpicture}
    \caption{Linea de transmisión en circuito abierto.}
    \label{fig:lt}
  \end{figure}


\begin{solution}
\begin{parts}
\part{La impedancia $Z_{ca}$ del circuito se calculará con la formula $Z_{in}$ para la cual se tiene que $Z_L = \infty$ y $Z_0 = 50\,\Omega$.}

\begin{align}
  Z_{ca} &=\frac{Z_{0} (Z_{l}+ jZ_{0}\tan(\beta l))}{(Z_{0}+jZ_{l}\tan(\beta l))}\\
  &= Z_{0} \frac{\del{ 1 + \frac{jZ_{0}\tan(\beta l)}{Z_{l}}}}{\del{(\frac{Z_{0}}{Z_{l}} + j\tan(\beta l)}}\\
  &= \frac{-jZ_{0}}{\tan\del{\frac{2\pi}{\lambda} \frac{5\lambda}{12\pi}}}\\
  &= \frac{-jZ_{0}}{\tan\del{\frac{5\pi}{6}}}\\
  &= \frac{j \sqrt{3}Z_{0}}{3}
\end{align}
\part{La impedancia equivalente en la linea de transmisión será la impedancia de circuito abierto en paralelo con la impedancia de la carga $Z_{eq} = Z_{ca}//Z_{l}$.}
En este caso se tiene que $Z_{l} = 50-50j$ y $Z_{ca} = j \sqrt{3}Z_{0}/3$ por lo que para conseguir $Z_{eq}$ se tiene que:
\begin{equation}
  Z_{eq} = \frac{Z_{ca}Z_{L}}{Z_{ca}+Z_{L}} = \frac{j \sqrt{3}Z_{0}/3(50-50j)}{j \sqrt{3}Z_{0}/3+50-50j}
\end{equation}
Esto nos dará como resultado
\begin{equation}
  Z_{eq} = 83.47 + 37.15j
\end{equation}
\part{Luego normalizando la impedancia se tiene $\frac{Z_{eq}}{40}$}
\begin{equation}
  Z_{eq} = 2.08 + 0.93j
\end{equation}
Luego calculando la admitancia para utilizar la carta smith se tiene:
\begin{equation}
  Y_{eq} = \frac{1}{Z_{eq}} = 0.4 - 0.2j
\end{equation}
Por lo que en la carta de smith se tiene que la parte real de la admitancia es $0.4$ y la parte imaginaria es $-0.2$ por lo que se tiene utilizando la carta smith conseguimos los siguientes valores para $l$, $l_s^{cc}$ y $l_s^{ca}$.
\begin{align}
  l &= 0.2\lambda\\
  l_s^{cc} &= 0.125\lambda\\
  l_s^{ca} &= 0.0.375\lambda
\end{align}
Recordar que cada uno de estos valores se obtiene de la carta de smith y corresponden a valores aproximados.






\end{parts}
\end{solution}


\end{questions}
\end{document}
