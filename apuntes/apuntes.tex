\documentclass[
  11pt,
  letterpaper,
  % addpoints,
  answers
]{exam}

\usepackage{../tarea}

\begin{document}
\begin{minipage}{0.42\textwidth}
    \includegraphics[width=\textwidth]{../fcfm_die}
\end{minipage}
\begin{minipage}{0.53\textwidth}
\begin{center} 
\large\textbf{Astro}  \\
\large\textbf{Resumen astro c2} \\
\normalsize Prof.~\\
\normalsize Prof.~\\
\normalsize Ayudantes: 
\end{center}
\end{minipage}

\vspace{0.5cm}
\noindent
\vspace{.85cm}




\begin{itemize}
    \item El  telescpoio en el espacio, JWST observa el infrarojo, por lo que miro longitudes de onda mas largas que el optico, por lo que es ideal para observar galaxias a alto redshift, desde Z>5, con Z el redshift $\lambda_{observado} = (1+Z)\lambda_{origen}$
    \item Galaxias: las que son suficientemente masivas deberían necesitar un agujero negro super masivo en su centro, pero no todas las galaxias lo necesitan.
    \item las galaxia estan formadas por polvo, en donde las espirales tienen un monton de polvo, ya que hay formación de estrellas, por lo que su espectro tiene un monton de lineas de emisión (que proviene de gas caliente, es decir zonas de formacion estelar), en donde más alto el peak más alto la formacion de estrellas. Pueden tener brazos espirales, y tienen estructura de discos
    \item hay otro tipo de galaxia, la elipitica, en donde su gracia, es que son super brillantes, y se les dice que son rojas y muertas, por que no hay formación estelar, las rojas viven mucho más tiempo que las axules, en donde sus espectros no hay lineas de emision, y se caracterizan por un corte a los 4000 amstrong, producto de absorción de metales. Sus estrellas estan enriquicidas en metales.
    \item Tecnicas de observación de galaxias: tenemos la espectroscopia, nos permite determinar los parametros de las galaxias cuanta luz me llega
    \item Tambien tenemos la fotometria en donde se usan filtros como el B, V, U, la fotometria es poco costosa, y tambien podemos estudiar la morfologia de la galaxia 
    \item Alma es radio-telescopio, donde se pueden observar galaxias y moleculas, carbono, etc.
    \item En resumen para estudiar diferentes tipos de galaxia tenemos diferentes metodos de observación y estudio de las galaxias como el VLT / MUSE, que utiliza espectroscopia de campo integral, donde tomo un espectro de todo el espacio que estoy observando
    \item Esquema de hubble desde elipticas a espirales, creemos que las galaxias parten siendo espirales y terminan siendo elipticas, en donde estas evolucionan en forma jerarquica, donde las galaxias chocan.
    \item Ciclo de bariones: un barion es de los protones para arriba, atomos moleculas metales etc, el ciclo de bariones es el ciclo de vida de los bariones, en donde se forman en las estrellas, y se liberan en las supernovas expulsando un monton de material enriquecido como carbono magnesio oxigeno, en donde los agujeros negros tambien pueden expulsar materiales en forma de jets violentamente, esto lo llamamos como vientos galacticos. Hay galaxias que se comen el material gaseoso que esta en el universo en el medio intergalactico
    \item En el ciclo de bariones tenemos los inflows: acreción de material gasesos, el cual su origen puede ser el IGM (medio intergalactico) es muy dificil detectar los inflows
    \item tenemos los outflows: origen en supernovas y AGN 
    \item Luego el material explusado puede volver a caer en la galaxia llamado reciclaje, este material es enriqucido, estos 3 procesos son denominados como el ciclo de bariones. Esto ocurre en el medio intergalactico o CGM, fuera del disco estelar, pero a una distancia suficiente para seguir ligado a las galaxias.
    \item Agujeros negros: en la relatividad general es facil describirlo.
    \item cuando las galaxias tienen un AGN como en un cuasar el espectro lo domina el cuasar, donde el espectro son lineas muy anchas con layman alpha.
    \item Cuando hay lentes de galaxias, se dice que es por un cumulo de galaxias, o cuando un cuasar se alinea con una galaxia, en donde la galaxia actuca como lente, luego si el espectro es el mismo, estamos en presencia de un cuasar.
    \item Time delays: observa variaciones temporales en cuasares lenceados, cuando vemos varios focos de luz de un cuasar, podemos ver como al tomar caminos distintos, llegan en tiempos distintos. Al medirlos podemos obtener cosas como la constante de Hubble, en donde $\Delta T \implies H_{0}^-1$, esto se mide con fotometria (estudiar time delays)
    \item bosques de layman alpha en donde las absorciones son producto de la absorción de luz del medio intergalactico, esto permite reconstruir el medio intergalactico mediante mapas, neccesitamos espectroscopia para estudiarlo, lineas de layman alpha 1215A, en donde nos tenemos que mover al redshit 2 para poder observarlos. $(1+Z)121.5nanometros$ 364,5 nanometros, y eso está en el optico.
    \item Para formar estrellas hay que estar muy helado ISM
    \item El ISM siempre se observa con radio telescopios, ya que es muy frio como el alma
    \item Evolucion estelar, las rojas gigantes son super frias, y son como el 90\% de la vida de las estrellas, luego el sol expulsará su material de a poquito y se convertira en una enana blanca. En las supernovas se forman objetos mas pesados, y se llama nucleosintesis estelar, en donde se forman elementos mas pesados que el hierro, en donde se forman elementos como el oro, plomo, etc.
    \item Se supone que no hay estrellas primitivas, y la unica indicación es que ya estan todas extintas y eran super masivas, por lo que su vida era muy corta.
    \item principio de exclusión de pauli, no puede haber dos fermiones en el mismo estado cuantico
    \item cuando las estrellas de neutrones expulsan campos electrogamneticos se llaman pulsares, y se pueden observar en el espectro de radio.
    \item para estudiar la via lactea y evadir el polvo nos vamos a la ondas de radio
    \item burbujas de fermin y burbujas de loop 1, son burbujas de gas caliente, se supone que es el recordatorio de que el agujero negro de nuestra galaxia alguna vez fue activo 
    \item como se observa que instrumentos se ocupan que se sabe y que no se sabe 
\end{itemize}










\end{document}
